\part{Осцилляции}
{\color{brown}Предполагается извлекать погрешности параметров осцилляций, порождаемые прочими неопределённостями расчёта: погрешностью спектра, профиля Земли и сечения взаимодействия (из-за неопределённости ядерной модели, структурных функций нуклонов и пр.).}

{\color{MYgreen}Необходимо оценить, какую погрешность имеют результаты современных статистических анализов данных экспериментов по изучению нейтринных осцилляций, насколько они различаются между собой и какую неопределённость это вносит в лептонные потоки, для чего нужно рассмотреть также анализы {\color{red}других} групп.}

В качестве параметров осцилляций взяты данные групп Valle \emph{et al.}~\cite{Tortola:2012te} (Табл.~\ref{Valle}) и Fogli \emph{et al.}~\cite{Fogli:2012ua} (Табл.~\ref{Fogli}) 2012 года. Таблицы~\ref{Valle}--\ref{Fogli} показывают серьёзное различие значений угла $\theta_{23}$.

\begin{table}[!ht]
\begin{center}
\caption{\label{Valle}Параметры осцилляций нейтрино в предположении {\color{blue}нормальной} и {\color{violet}обратной} иерархии масс согласно анализу группы Valle~\cite{Tortola:2012te}}
\vspace{2mm}
\begin{tabular}{|c|c|c|c|c|}
\hline
parameter	&\multicolumn{1}{|c}{best fit}	&\multicolumn{1}{|c}{$1\sigma$}	&\multicolumn{1}{|c}{$2\sigma$}	&\multicolumn{1}{|c|}{$3\sigma$}\\
\hline
$\phantom{|}\Delta{m}^{2}_{21}\phantom{|}\: [10^{-5}\text{eV}^{2}]$	&$7.62$	&$7.43-7.81$	&$7.27-8.01$	&$7.12-8.20$\\
\hline
$|\Delta{m}^{2}_{31}|\: [10^{-3}\text{eV}^{2}]$
	&\begin{tabular}{c}
		{\color{blue}$2.55$}\\
		{\color{violet}$2.43$}
	\end{tabular}	
	&\begin{tabular}{c}
		{\color{blue}$2.46-2.61$}\\
		{\color{violet}$2.37-2.50$}
	\end{tabular}
	&\begin{tabular}{c}
		{\color{blue}$2.38-2.68$}\\
		{\color{violet}$2.29-2.58$}
	\end{tabular}	
	&\begin{tabular}{c}
		{\color{blue}$2.31-2.74$}\\
		{\color{violet}$2.21-2.64$}
	\end{tabular}\\[3mm]
\hline
$\sin^{2}\theta_{12}\: [10^{-1}]$	&$3.20$	&$3.03-3.36$	&$2.90-3.50$	&$2.70-3.70$\\
\hline
$\sin^{2}\theta_{13}\: [10^{-2}]$	
	&\begin{tabular}{c}
		{\color{blue}$2.46$}\\
		{\color{violet}$2.50$}
	\end{tabular}
	&\begin{tabular}{c}
		{\color{blue}$2.18-2.75$}\\
		{\color{violet}$2.23-2.76$}
	\end{tabular}
	&\begin{tabular}{c}
		{\color{blue}$1.90-3.00$}\\
		{\color{violet}$2.00-3.00$}
	\end{tabular}	&$1.70-3.30$\\[3mm]
\hline
$\sin^{2}\theta_{23}\: [10^{-1}]$	
	&\begin{tabular}{c}
		{\color{blue}$(4.27)\: 6.13$}\\
		{\color{violet}$\phantom{(4.27)\: }6.00$}
	\end{tabular}
	&\begin{tabular}{c}
		{\color{blue}$4.00-4.61\cup5.73-6.35$}\\
		{\color{violet}$5.69-6.26$}
	\end{tabular}
	&\begin{tabular}{c}
		{\color{blue}$3.80-6.60$}\\
		{\color{violet}$3.90-6.50$}
	\end{tabular}
	&\begin{tabular}{c}
		{\color{blue}$3.60-6.80$}\\
		{\color{violet}$3.70-6.70$}
	\end{tabular}\\[3mm]
\hline
$\delta/\pi$	
	&\begin{tabular}{c}
		{\color{blue}$\phantom{-}0.80$}\\
		{\color{violet}$-0.03$}
	\end{tabular}
	&$0-2$	&$0-2$	&$0-2$\\[3mm]
\hline
\end{tabular}
\end{center}
\end{table}

\begin{table}[!ht]
\begin{center}
\caption{\label{Fogli}Параметры осцилляций нейтрино в предположении {\color{blue}нормальной} и {\color{violet}обратной} иерархии масс согласно анализу группы Fogli~\cite{Fogli:2012ua}. Видно серьёзное отличие значения угла $\theta_{23}$ от результата группы Valle}
\vspace{2mm}
\begin{tabular}{|c|c|c|c|c|}
\hline
parameter	&\multicolumn{1}{|c}{best fit}	&\multicolumn{1}{|c}{$1\sigma$}	&\multicolumn{1}{|c}{$2\sigma$}	&\multicolumn{1}{|c|}{$3\sigma$}\\
\hline
$\Delta{m}^{2}_{21}=\phantom{|}\delta{m}^{2}\phantom{|}\: [10^{-5}\text{eV}^{2}]$	&$7.54$	&$7.32-7.80$	&$7.15-8.00$	&$6.99-8.18$\\
\hline
\begin{tabular}{c}
	$m_{3}^{2}-(m_{1}^{2}+m_{2}^{2})/2=$\\
	$=|\Delta{m}^{2}|\: [10^{-3}\text{eV}^{2}]$
\end{tabular}
	&\begin{tabular}{c}
		{\color{blue}$2.43$}\\
		{\color{violet}$2.42$}
	\end{tabular}
	&\begin{tabular}{c}
		{\color{blue}$2.33-2.49$}\\
		{\color{violet}$2.31-2.49$}
	\end{tabular}
	&\begin{tabular}{c}
		{\color{blue}$2.27-2.55$}\\
		{\color{violet}$2.26-2.53$}
	\end{tabular}	
	&\begin{tabular}{c}
		{\color{blue}$2.19-2.62$}\\
		{\color{violet}$2.17-2.61$}
	\end{tabular}\\[3mm]
\hline
$\sin^{2}\theta_{12}\: [10^{-1}]$	&$3.07$	&$2.91-3.25$	&$2.75-3.42$	&$2.59-3.59$\\
\hline
$\sin^{2}\theta_{13}\: [10^{-2}]$	
	&\begin{tabular}{c}
		{\color{blue}$2.41$}\\
		{\color{violet}$2.44$}
	\end{tabular}
	&\begin{tabular}{c}
		{\color{blue}$2.16-2.66$}\\
		{\color{violet}$2.19-2.67$}
	\end{tabular}
	&\begin{tabular}{c}
		{\color{blue}$1.93-2.90$}\\
		{\color{violet}$1.94-2.91$}
	\end{tabular}	
	&\begin{tabular}{c}
		{\color{blue}$1.69-3.13$}\\
		{\color{violet}$1.71-3.15$}
	\end{tabular}\\[3mm]
\hline
$\sin^{2}\theta_{23}\: [10^{-1}]$	
	&\begin{tabular}{c}
		{\color{blue}$3.86$}\\
		{\color{violet}$3.92$}
	\end{tabular}
	&\begin{tabular}{c}
		{\color{blue}$3.65-4.10$}\\
		{\color{violet}$3.70-4.31$}
	\end{tabular}
	&\begin{tabular}{c}
		{\color{blue}$3.48-4.48$}\\
		{\color{violet}$3.53-4.84\cup5.43-6.41$}
	\end{tabular}	
	&\begin{tabular}{c}
		{\color{blue}$3.31-6.37$}\\
		{\color{violet}$3.35-6.63$}
	\end{tabular}\\[3mm]
\hline
$\delta/\pi$
	&\begin{tabular}{c}
		{\color{blue}$1.08$}\\
		{\color{violet}$1.09$}
	\end{tabular}
	&\begin{tabular}{c}
		{\color{blue}$0.77-1.36$}\\
		{\color{violet}$0.83-1.47$}
	\end{tabular}
	&$0-2$	&$0-2$\\[3mm]
\hline
\end{tabular}
\end{center}
\end{table}

{\color{MYgreen}Необходимо выбрать базис (с $\Delta{m}^{2}_{31}$ или с $\Delta{m}^{2}$) и пересчитать для него погрешности.} Для оценки влияния статистических ошибок на интегральные величины случайным образом выбрать множество точек из разрешённого гиперобъёма значений параметров и построить функции наибольшего и наименьшего значения результата в каждой точке.

Рисунки~\ref{vacnFn_ee_10vs50} и~\ref{vacnFn_ee_50vs250} представляют собой карты значений функции $P_{ee}(E_{\nu},\theta)$ в области $E_{\nu}\in[10^{-2},10^{2}]$~ГэВ и $\theta\in[\pi/2,\pi]$ в случае вакуумных осцилляций с параметрами анализа Fogli~\cite{Fogli:2012ua} в предположении нормальной иерархии масс, построенные с разной степенью подробности таблиц. {\color{magenta}Осциллограммы построены для всех переходов, типов нейтрино, иерархий масс, включённых наборов параметров.}

\begin{figure}[!ht]
\includegraphics[width=0.33\textwidth]{../pictures/vacnFn_ee_100G.eps}
\includegraphics[width=0.33\textwidth]{../pictures/vacnFn_ee_100_25G.eps}
\includegraphics[width=0.33\textwidth]{../pictures/vacnFn_ee_2500G.eps}
\caption{Двумерные графики вероятности выживания электронного нейтрино в случае вакуумных осцилляций с параметрами анализа Fogli~\cite{Fogli:2012ua} в предположении нормальной иерархии масс нейтрино в зависимости от энергии на выбранном интервале и косинуса надирного угла на всём промежутке прохождения Земли. \textit{Построено с помощью скрипта \texttt{ogramms.c} при выбранном параметре \texttt{hf=0} по данным таблиц \texttt{vacnFn\_ee.dat}, насчитанных программой \texttt{oscvac3}. На левом графике взято $10\times10$ точек, на центральном --- столько же, но применена пятикратная интерполяция, на правом --- $50\times50$.} Видно разительное отличие центрального графика от правого. {\color{red}Не настраивается диапазон шкалы.} {\color{magenta}Неподходящий для печати формат}}
\label{vacnFn_ee_10vs50}
\end{figure}
\begin{figure}[!ht]
\vspace*{-0.5cm}
\includegraphics[width=0.33\textwidth]{../pictures/vacnFn_ee_2500G.eps}
\includegraphics[width=0.33\textwidth]{../pictures/vacnFn_ee_2500_25G.eps}
\includegraphics[width=0.33\textwidth]{../pictures/vacnFn_ee_62500H.eps}
\caption{Карты вероятности выживания электронного нейтрино в случае вакуумных осцилляций с параметрами анализа Fogli~\cite{Fogli:2012ua} в предположении нормальной иерархии масс нейтрино в зависимости от энергии на выбранном интервале и косинуса надирного угла на всём промежутке прохождения Земли. \textit{Построено с помощью скрипта \texttt{ogramms.c} при выбранном параметре \texttt{hf=0} для графиков слева и в центре и \texttt{hf=1} для гистограммы справа по данным таблиц \texttt{vacnFn\_ee.dat}, насчитанных программой \texttt{oscvac3}. На первом графике взято $50\times50$ точек, на втором --- столько же, но применена пятикратная интерполяция. Невозможно построить график для $250\times250$ точек, поэтому приводится соответствующая гистограмма.} Видно значительное различие между центральным и правым графиками. {\color{red}Не настраивается диапазон шкалы.} {\color{magenta}Неподходящий для печати формат}}
\label{vacnFn_ee_50vs250}
\end{figure}

Рисунок~\ref{vacnFn_ee_10vs50} показывает, что 10 точек на измерение совершенно не достаточно для того, чтобы охарактеризовать функцию $P_{\alpha\beta}(E_{\nu},\theta)$. Рисунок~\ref{vacnFn_ee_50vs250} демонстрирует, что взятие 50 точек на измерение даёт представление о поведении функции $P_{\alpha\beta}(E_{\nu},\theta)$ в области высоких энергий нейтрино, где осцилляции слабы, однако этого количества недостаточно для описания функции в области низких энергий ($\lesssim{}0.3$~ГэВ).

\begin{figure}[!ht]
\includegraphics[width=0.5\textwidth]{../pictures/vacnFn_ee_2500H.eps}
\includegraphics[width=0.5\textwidth]{../pictures/vacnFn_ee_2500G.eps}
\caption{Карты вероятности выживания электронного нейтрино в случае вакуумных осцилляций с параметрами анализа Fogli~\cite{Fogli:2012ua} в предположении нормальной иерархии масс нейтрино в зависимости от энергии на выбранном интервале и косинуса надирного угла на всём промежутке прохождения Земли. \textit{Построено с помощью скрипта \texttt{ogramms.c} при выбранном параметре \texttt{hf=1} для гистограммы слева и \texttt{hf=0} для графика справа по данным таблиц \texttt{vacnFn\_ee.dat}, насчитанных программой \texttt{oscvac3} (взято $50\times50$ точек).} Заметно различие между картами, в частности, {\color{red}в кривизне изгиба полос}. {\color{red}Не настраивается диапазон шкалы.} {\color{magenta}Неподходящий для печати формат}}
\label{vacnFn_ee_HvsG}
\end{figure}

Рисунок~\ref{vacnFn_ee_HvsG} показывает различие между картами функции $P_{\alpha\beta}(E_{\nu},\theta)$, построенными в виде двумерной гистограммы и двумерного графика. Помимо прочего, оно проявляется {\color{red}в кривизне изгиба полос}, что заставляет относиться к двумерным графикам с недоверием. {\color{blue}В качестве формы представления осциллограмм выбраны двумерные гистограммы.}

На рисунке~\ref{vacn3DFn} представлены подробные вакуумные осциллограммы для всех переходов нейтрино, вычисленные с параметрами анализа Fogli~\cite{Fogli:2012ua} в предположении нормальной иерархии масс нейтрино для всех переходов нейтрино.
\clearpage
\begin{figure}[!ht]
\begin{comment}
\includegraphics[width=0.33\textwidth]{../pictures/3D_1562500H/vacnFn_ee.eps}
\includegraphics[width=0.33\textwidth]{../pictures/3D_1562500H/vacnFn_em.eps}
\includegraphics[width=0.33\textwidth]{../pictures/3D_1562500H/vacnFn_et.eps}
\includegraphics[width=0.33\textwidth]{../pictures/3D_1562500H/vacnFn_me.eps}
\includegraphics[width=0.33\textwidth]{../pictures/3D_1562500H/vacnFn_mm.eps}
\includegraphics[width=0.33\textwidth]{../pictures/3D_1562500H/vacnFn_mt.eps}
\includegraphics[width=0.33\textwidth]{../pictures/3D_1562500H/vacnFn_te.eps}
\includegraphics[width=0.33\textwidth]{../pictures/3D_1562500H/vacnFn_tm.eps}
\includegraphics[width=0.33\textwidth]{../pictures/3D_1562500H/vacnFn_tt.eps}
\end{comment}
\includegraphics[width=\textwidth]{../pictures/3D_1562500H/vacnFn_1562500H.eps}
\caption{Вакуумные осциллограммы нейтрино, вычисленные с параметрами анализа Fogli~\cite{Fogli:2012ua} в предположении нормальной иерархии масс нейтрино в зависимости от энергии на выбранном интервале и косинуса надирного угла на всём промежутке прохождения Земли. \textit{Построено с помощью скрипта \texttt{ogramms.c} при выбранном параметре \texttt{hf=1} по данным таблиц \texttt{vacnFn*.dat}, насчитанных программой \texttt{oscvac3} (взято $1250\times1250$ точек).} Видна нарушенная симметрия между вероятностями $P_{\alpha\beta}$ и $P_{\beta\alpha}$. {\color{magenta}Неподходящий для печати формат, очень тяжёлый даже для вывода на экран, особенно в составе документа, {\color{red}вследствие чего вставлена "репродукция"{}}}}
\label{vacn3DFn}
\end{figure}

На рисунках~\ref{vacn3DF} и~\ref{vacn3Dn} представлены гистограммы функции $P_{ee}(E_{\nu},\theta)$ в области $E_{\nu}\in[10^{-2},10^{2}]$ ГэВ и $\theta\in[\pi/2,\pi]$ в случае вакуумных осцилляций. Для построения рисунка~\ref{vacn3DF} использованы параметры анализа Fogli~\cite{Fogli:2012ua}. Рисунок~\ref{vacn3Dn} получен в предположении нормальной иерархии масс нейтрино. Видно, что выбор комплекта значений параметров и варианта иерархии оказывает небольшое влияние на вероятности осцилляций.
\clearpage
\begin{figure}[!ht]
\includegraphics[width=0.5\textwidth]{../pictures/vacnFn_ee_2500H.eps}
\includegraphics[width=0.5\textwidth]{../pictures/vacnFi_ee_2500H.eps}
\caption{Гистограммы вероятностей выживания электронного нейтрино в зависимости от энергии на выбранном интервале и косинуса надирного угла на всём промежутке прохождения Земли в случае вакуумных осцилляций с параметрами анализа Fogli~\cite{Fogli:2012ua} в предположении нормальной (слева) и обратной (справа) иерархии масс нейтрино. \textit{Построено с помощью скрипта \texttt{ogramms.c} при выбранном параметре \texttt{hf=1} по данным таблиц \texttt{vacnF*\_ee.dat}, насчитанных программой \texttt{oscvac3} (взято $50\times50$ точек).} Видна небольшая разница между осциллограммами. {\color{magenta}Неподходящий для печати формат}}
\label{vacn3DF}
\end{figure}

\begin{figure}[!ht]
\includegraphics[width=0.5\textwidth]{../pictures/vacnFn_ee_2500H.eps}
\includegraphics[width=0.5\textwidth]{../pictures/vacnVn_ee_2500H.eps}
\caption{Гистограммы вероятностей выживания электронного нейтрино в зависимости от энергии на выбранном интервале и косинуса надирного угла на всём промежутке прохождения Земли в случае вакуумных осцилляций с параметрами анализа Fogli~\cite{Fogli:2012ua} (слева) и Valle~\cite{Tortola:2012te} (справа) в предположении нормальной иерархии масс нейтрино. \textit{Построено с помощью скрипта \texttt{ogramms.c} по данным таблиц \texttt{vacn*n\_ee.dat}, насчитанных программой \texttt{oscvac3} при выбранном параметре \texttt{hf=1} (взято $50\times50$ точек).} Видна небольшая разница между осциллограммами. {\color{magenta}Неподходящий для печати формат}}
\label{vacn3Dn}
\end{figure}

\begin{figure}[!ht]
\includegraphics[width=0.5\textwidth]{../pictures/3D_2500H/vacnFn_em.eps}
\includegraphics[width=0.5\textwidth]{../pictures/3D_2500H/vacnFn_me.eps}
\includegraphics[width=0.5\textwidth]{../pictures/3D_2500H/vacaFn_em.eps}
\includegraphics[width=0.5\textwidth]{../pictures/3D_2500H/vacaFn_me.eps}
\caption{Гистограммы вероятностей перехода электронного нейтрино (сверху) и антинейтрино (снизу) в мюонное (слева) и наоборот (справа) в зависимости от энергии на выбранном интервале и косинуса надирного угла на всём промежутке прохождения Земли в случае вакуумных осцилляций с параметрами анализа Fogli~\cite{Fogli:2012ua} в предположении нормальной иерархии масс нейтрино. \textit{Построено с помощью скрипта \texttt{ogramms.c} по данным таблиц \texttt{vac*Fn*.dat}, насчитанных программой \texttt{oscvac3} при выбранном параметре \texttt{hf=1} (взято $50\times50$ точек).} Гистограмма $P^{\nu}_{e\mu}$ совпадает с $P^{\bar\nu}_{\mu{}e}$, $P^{\nu}_{\mu{}e}$ --- с $P^{\bar\nu}_{e\mu}$. {\color{magenta}Неподходящий для печати формат}}
\label{vac3DFn}
\end{figure}
\newpage
В вакуумном случае $P^{\nu}_{\alpha\beta}=P^{\bar\nu}_{\beta\alpha}$, что подтверждает рисунок~\ref{vac3DFn}. На рисунке~\ref{vacn3DFn_dCP=0} видно, что $P^{\nu}_{\alpha\beta}=P^{\nu}_{\beta\alpha}$, если положить $\delta_{CP}=0$. Если занулить все углы смешивания, осцилляции исчезнут. \textsf{Это объясняется тем, что все матрицы поворота базиса состояний нейтрино превращаются в единичные, а дираковская матрица нарушения CP-симметрии сокращается своим сопряжённым.} Если занулить {\color{red}все разности масс} --- тоже {\color{red}(а у меня выходит, что вероятности выживаний вообще не зависят от разностей масс)}.
\clearpage
\begin{figure}[!ht]
\includegraphics[width=0.115\textwidth]{../pictures/3D_2500H/vacnFn_ee.eps}\hspace*{-2mm}
\includegraphics[width=0.115\textwidth]{../pictures/3D_2500H/vacnFn_em.eps}\hspace*{-2mm}
\includegraphics[width=0.115\textwidth]{../pictures/3D_2500H/vacnFn_et.eps}\hspace*{-2mm}
\includegraphics[width=0.115\textwidth]{../pictures/3D_2500H/vacnFn_me.eps}\hspace*{-2mm}
\includegraphics[width=0.115\textwidth]{../pictures/3D_2500H/vacnFn_mm.eps}\hspace*{-2mm}
\includegraphics[width=0.115\textwidth]{../pictures/3D_2500H/vacnFn_mt.eps}\hspace*{-2mm}
\includegraphics[width=0.115\textwidth]{../pictures/3D_2500H/vacnFn_te.eps}\hspace*{-2mm}
\includegraphics[width=0.115\textwidth]{../pictures/3D_2500H/vacnFn_tm.eps}\hspace*{-2mm}
\includegraphics[width=0.115\textwidth]{../pictures/3D_2500H/vacnFn_tt.eps}
\caption{Вакуумные осциллограммы нейтрино, вычисленные с параметрами анализа Fogli~\cite{Fogli:2012ua} в предположении нормальной иерархии масс нейтрино в зависимости от энергии на выбранном интервале и косинуса надирного угла на всём промежутке прохождения Земли. \textit{Построено с помощью скрипта \texttt{ogramms.c} по данным таблиц \texttt{vacnFn*.dat}, насчитанных программой \texttt{oscvac3} (взято $50\times50$ точек).} Ниже приводятся аналогичные осциллограммы, при расчёте которых один из параметров положен равным нулю. {\color{magenta}Неподходящий для печати формат}}
\label{vacn3DFn_ex}
\end{figure}
\begin{figure}[!ht]
\vspace*{-0.75cm}
\includegraphics[width=0.115\textwidth]{../pictures/3D_dCP=0/vacnFn_ee.eps}\hspace*{-2mm}
\includegraphics[width=0.115\textwidth]{../pictures/3D_dCP=0/vacnFn_em.eps}\hspace*{-2mm}
\includegraphics[width=0.115\textwidth]{../pictures/3D_dCP=0/vacnFn_et.eps}\hspace*{-2mm}
\includegraphics[width=0.115\textwidth]{../pictures/3D_dCP=0/vacnFn_me.eps}\hspace*{-2mm}
\includegraphics[width=0.115\textwidth]{../pictures/3D_dCP=0/vacnFn_mm.eps}\hspace*{-2mm}
\includegraphics[width=0.115\textwidth]{../pictures/3D_dCP=0/vacnFn_mt.eps}\hspace*{-2mm}
\includegraphics[width=0.115\textwidth]{../pictures/3D_dCP=0/vacnFn_te.eps}\hspace*{-2mm}
\includegraphics[width=0.115\textwidth]{../pictures/3D_dCP=0/vacnFn_tm.eps}\hspace*{-2mm}
\includegraphics[width=0.115\textwidth]{../pictures/3D_dCP=0/vacnFn_tt.eps}
\caption{$\delta_{CP}=0$. Совпадают гистограммы для $P^{\nu}_{e\mu}$ и $P^{\nu}_{\mu{}e}$, $P^{\nu}_{e\tau}$ и $P^{\nu}_{\tau{}e}$, $P^{\nu}_{\mu\tau}$ и $P^{\nu}_{\tau\mu}$}
\label{vacn3DFn_dCP=0}
\end{figure}
\begin{figure}[!ht]
\vspace*{-0.75cm}
\includegraphics[width=0.115\textwidth]{../pictures/3D_m12=0/vacnFn_ee.eps}\hspace*{-2mm}
\includegraphics[width=0.115\textwidth]{../pictures/3D_m12=0/vacnFn_em.eps}\hspace*{-2mm}
\includegraphics[width=0.115\textwidth]{../pictures/3D_m12=0/vacnFn_et.eps}\hspace*{-2mm}
\includegraphics[width=0.115\textwidth]{../pictures/3D_m12=0/vacnFn_me.eps}\hspace*{-2mm}
\includegraphics[width=0.115\textwidth]{../pictures/3D_m12=0/vacnFn_mm.eps}\hspace*{-2mm}
\includegraphics[width=0.115\textwidth]{../pictures/3D_m12=0/vacnFn_mt.eps}\hspace*{-2mm}
\includegraphics[width=0.115\textwidth]{../pictures/3D_m12=0/vacnFn_te.eps}\hspace*{-2mm}
\includegraphics[width=0.115\textwidth]{../pictures/3D_m12=0/vacnFn_tm.eps}\hspace*{-2mm}
\includegraphics[width=0.115\textwidth]{../pictures/3D_m12=0/vacnFn_tt.eps}
\caption{$\Delta{m_{12}}=0$. $P^{\nu}_{e\mu}=P^{\nu}_{\mu{}e}=0$, $P^{\nu}_{e\tau}=P^{\nu}_{\tau{}e}$, $P^{\nu}_{e\tau}\simeq{}P^{\nu}_{\tau{}e}$. $P^{\nu}_{\mu\mu}\approx{}P^{\nu}_{\tau\tau}$}
\label{vacn3DFn_m12=0}
\end{figure}
\begin{figure}[!ht]
\vspace*{-0.75cm}
\includegraphics[width=0.115\textwidth]{../pictures/3D_m13=0/vacnFn_ee.eps}\hspace*{-2mm}
\includegraphics[width=0.115\textwidth]{../pictures/3D_m13=0/vacnFn_em.eps}\hspace*{-2mm}
\includegraphics[width=0.115\textwidth]{../pictures/3D_m13=0/vacnFn_et.eps}\hspace*{-2mm}
\includegraphics[width=0.115\textwidth]{../pictures/3D_m13=0/vacnFn_me.eps}\hspace*{-2mm}
\includegraphics[width=0.115\textwidth]{../pictures/3D_m13=0/vacnFn_mm.eps}\hspace*{-2mm}
\includegraphics[width=0.115\textwidth]{../pictures/3D_m13=0/vacnFn_mt.eps}\hspace*{-2mm}
\includegraphics[width=0.115\textwidth]{../pictures/3D_m13=0/vacnFn_te.eps}\hspace*{-2mm}
\includegraphics[width=0.115\textwidth]{../pictures/3D_m13=0/vacnFn_tm.eps}\hspace*{-2mm}
\includegraphics[width=0.115\textwidth]{../pictures/3D_m13=0/vacnFn_tt.eps}
\caption{$\Delta{m_{13}}=0$. $P^{\nu}_{e\mu}\simeq{}P^{\nu}_{\mu{}e}$, $P^{\nu}_{e\tau}\simeq{}P^{\nu}_{\tau{}e}$, $P^{\nu}_{e\tau}=P^{\nu}_{\tau{}e}$}
\label{vacn3DFn_m13=0}
\end{figure}
\begin{figure}[!ht]
\vspace*{-0.75cm}
\includegraphics[width=0.115\textwidth]{../pictures/3D_m23=0/vacnFn_ee.eps}\hspace*{-2mm}
\includegraphics[width=0.115\textwidth]{../pictures/3D_m23=0/vacnFn_em.eps}\hspace*{-2mm}
\includegraphics[width=0.115\textwidth]{../pictures/3D_m23=0/vacnFn_et.eps}\hspace*{-2mm}
\includegraphics[width=0.115\textwidth]{../pictures/3D_m23=0/vacnFn_me.eps}\hspace*{-2mm}
\includegraphics[width=0.115\textwidth]{../pictures/3D_m23=0/vacnFn_mm.eps}\hspace*{-2mm}
\includegraphics[width=0.115\textwidth]{../pictures/3D_m23=0/vacnFn_mt.eps}\hspace*{-2mm}
\includegraphics[width=0.115\textwidth]{../pictures/3D_m23=0/vacnFn_te.eps}\hspace*{-2mm}
\includegraphics[width=0.115\textwidth]{../pictures/3D_m23=0/vacnFn_tm.eps}\hspace*{-2mm}
\includegraphics[width=0.115\textwidth]{../pictures/3D_m23=0/vacnFn_tt.eps}
\caption{$\Delta{m_{23}}=0$. $P^{\nu}_{e\mu}\simeq{}P^{\nu}_{\mu{}e}$, $P^{\nu}_{e\tau}=P^{\nu}_{\tau{}e}$, $P^{\nu}_{e\tau}\simeq{}P^{\nu}_{\tau{}e}$}
\label{vacn3DFn_m23=0}
\end{figure}
\begin{figure}[!ht]
\vspace*{-0.75cm}
\includegraphics[width=0.115\textwidth]{../pictures/3D_s12=0/vacnFn_ee.eps}\hspace*{-2mm}
\includegraphics[width=0.115\textwidth]{../pictures/3D_s12=0/vacnFn_em.eps}\hspace*{-2mm}
\includegraphics[width=0.115\textwidth]{../pictures/3D_s12=0/vacnFn_et.eps}\hspace*{-2mm}
\includegraphics[width=0.115\textwidth]{../pictures/3D_s12=0/vacnFn_me.eps}\hspace*{-2mm}
\includegraphics[width=0.115\textwidth]{../pictures/3D_s12=0/vacnFn_mm.eps}\hspace*{-2mm}
\includegraphics[width=0.115\textwidth]{../pictures/3D_s12=0/vacnFn_mt.eps}\hspace*{-2mm}
\includegraphics[width=0.115\textwidth]{../pictures/3D_s12=0/vacnFn_te.eps}\hspace*{-2mm}
\includegraphics[width=0.115\textwidth]{../pictures/3D_s12=0/vacnFn_tm.eps}\hspace*{-2mm}
\includegraphics[width=0.115\textwidth]{../pictures/3D_s12=0/vacnFn_tt.eps}
\caption{$\theta_{12}=0$. $P^{\nu}_{e\mu}=P^{\nu}_{\mu{}e}=0$, $P^{\nu}_{e\tau}=P^{\nu}_{\tau{}e}$, $P^{\nu}_{e\tau}=P^{\nu}_{\tau{}e}$. $P^{\nu}_{\mu\mu}\approx{}P^{\nu}_{\tau\tau}$}
\label{vacn3DFn_s12=0}
\end{figure}
\begin{figure}[!ht]
\vspace*{-0.75cm}
\includegraphics[width=0.115\textwidth]{../pictures/3D_s13=0/vacnFn_ee.eps}\hspace*{-2mm}
\includegraphics[width=0.115\textwidth]{../pictures/3D_s13=0/vacnFn_em.eps}\hspace*{-2mm}
\includegraphics[width=0.115\textwidth]{../pictures/3D_s13=0/vacnFn_et.eps}\hspace*{-2mm}
\includegraphics[width=0.115\textwidth]{../pictures/3D_s13=0/vacnFn_me.eps}\hspace*{-2mm}
\includegraphics[width=0.115\textwidth]{../pictures/3D_s13=0/vacnFn_mm.eps}\hspace*{-2mm}
\includegraphics[width=0.115\textwidth]{../pictures/3D_s13=0/vacnFn_mt.eps}\hspace*{-2mm}
\includegraphics[width=0.115\textwidth]{../pictures/3D_s13=0/vacnFn_te.eps}\hspace*{-2mm}
\includegraphics[width=0.115\textwidth]{../pictures/3D_s13=0/vacnFn_tm.eps}\hspace*{-2mm}
\includegraphics[width=0.115\textwidth]{../pictures/3D_s13=0/vacnFn_tt.eps}
\caption{$\theta_{13}=0$. $P^{\nu}_{e\mu}=P^{\nu}_{\mu{}e}$, $P^{\nu}_{e\tau}=P^{\nu}_{\tau{}e}$, $P^{\nu}_{e\tau}=P^{\nu}_{\tau{}e}$}
\label{vacn3DFn_s13=0}
\end{figure}
\begin{figure}[!ht]
\vspace*{-0.75cm}
\includegraphics[width=0.115\textwidth]{../pictures/3D_s23=0/vacnFn_ee.eps}\hspace*{-2mm}
\includegraphics[width=0.115\textwidth]{../pictures/3D_s23=0/vacnFn_em.eps}\hspace*{-2mm}
\includegraphics[width=0.115\textwidth]{../pictures/3D_s23=0/vacnFn_et.eps}\hspace*{-2mm}
\includegraphics[width=0.115\textwidth]{../pictures/3D_s23=0/vacnFn_me.eps}\hspace*{-2mm}
\includegraphics[width=0.115\textwidth]{../pictures/3D_s23=0/vacnFn_mm.eps}\hspace*{-2mm}
\includegraphics[width=0.115\textwidth]{../pictures/3D_s23=0/vacnFn_mt.eps}\hspace*{-2mm}
\includegraphics[width=0.115\textwidth]{../pictures/3D_s23=0/vacnFn_te.eps}\hspace*{-2mm}
\includegraphics[width=0.115\textwidth]{../pictures/3D_s23=0/vacnFn_tm.eps}\hspace*{-2mm}
\includegraphics[width=0.115\textwidth]{../pictures/3D_s23=0/vacnFn_tt.eps}
\caption{$\theta_{23}=0$. $P^{\nu}_{e\mu}=P^{\nu}_{\mu{}e}$, $P^{\nu}_{e\tau}=P^{\nu}_{\tau{}e}$, $P^{\nu}_{e\tau}=P^{\nu}_{\tau{}e}=0$}
\label{vacn3DFn_s23=0}
\end{figure}

\begin{comment}
\begin{figure}[!ht]
\includegraphics[width=0.33\textwidth]{../pictures/vac3an_aa.eps}
\includegraphics[width=0.33\textwidth]{../pictures/vac3an_ab.eps}
\includegraphics[width=0.33\textwidth]{../pictures/vac3an_ba.eps}
\caption{vac3an}
\label{vac3an}
\end{figure}
\end{comment}
