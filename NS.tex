\part{Спектры нейтрино}
\section{Спектры атмосферных нейтрино}
{\color{MYgreen}Необходимо оценить, насколько различаются между собой современные низкоэнергетические спектры атмосферных нейтрино и какую неопределённость эта разница вносит в лептонные потоки, для чего нужно рассмотреть также спектры Bartol (Гайссер-Станев \emph{et al.}) и FLUKA (Баттистони \emph{et al.}). Также нужен высокоэнергетический <<хвост>>, для чего подойдут спектры Синеговского, а возможно, CORT, в котором достаточно аккуратно учтены все важные эффекты, за исключением трёхмерности каскадов, которые не важны при высоких энергиях. Низко- и высокоэнергетические части спектра должны быть гладко сшиты. {\color{brown}Все опции спектров автоматизированно включать в CORTout.}}

\subsection{Низкоэнергетичные спектры}
\subsubsection{Honda11}
{\color{blue}Взят спектр<<Honda11>>}~\cite{Honda:2011nf}: 3D Монте-Карло расчёт потоков обычных атмосферных нейтрино Honda \emph{et al.} 2011 года. \textsf{3D-эффекты обусловлены несовпадением направлений первичной и вторичной частиц.} {\color{brown}Можно модифицировать CORTout, введя поправку на 3D, для чего потребуется найти и параметризовать относительный 3D-эффект.} \textsf{Из-за интерференции с геомагнитными поправками эффект будет немного отличаться для разных детекторов.} Поскольку 3D-расчёт имеет смысл только для низких энергий, на самом деле при некоторой энергии он сшивается с одномерным. {\color{brown}Основан на модели ядерных взаимодействий JAM и модифицированном пакете DPMJET-III.} Как \cite{Honda:2011nf}, так и предыдущая статья~\cite{Sanuki:2006yd}, в основном, описывают расчёт не с нуля, а по сравнению с более ранним. В другой предшествующей статье~\cite{Honda:2006qj} приводится  чуть более широкая по энергиям иллюстрация --- Рис.~\ref{comparison} показывает, что она воспроизводится нами по таблицам.
\begin{figure}[!ht]
\begin{center}
\includegraphics[width=0.45\textwidth]{../pictures/original.eps}
\includegraphics[width=0.45\textwidth]{../pictures/duplicate.eps}
\end{center}
\caption{Воспроизведение иллюстрации из статьи~\cite{Honda:2006qj}}
\label{comparison}
\end{figure}

На странице http://icrr.u-tokyo.ac.jp/$\sim$mhonda/nflx2011/index.html выложены таблицы потоков нейтрино в зависимости от энергии $E_{\nu}$ в промежутке $[10^{-1},10^{4}]$~ГэВ, косинуса зенитного угла $\theta$ от -1 до 1 по бинам 0.1 и азимутального угла $\varphi$ от $0^{\circ}$ до $360^{\circ}$ по бинам $30^{\circ}$, а также усреднённые по азимутальному углу с учётом и без учёта земной толщи над детектором и усреднённые по всем направлениям без учёта толщи для обсерваторий в Kamioka, Gran Sasso, Sudbury, Soudan, Frejus, INO в минимумах и максимумах солнечной активности.

\newpage
{\color{blue}Взята таблица для солнечного максимума без учёта горы над детектором для Kamioka в зависимости от азимутального угла.} Процедура \texttt{AN\_Honda11}, суммируя потоки по всем бинам азимутального угла и усредняя делением на число бинов, строит сплайны для полученного потока в зависимости от энергии и зенитного угла. {\color{MYgreen}Если зависимость от азимутального угла использоваться не будет, использовать авторское усреднение. В противном случае понадобится трёхмерный сплайн. Необходимо реализовать расчёт для любого момента солнечного цикла. Оценить, насколько сильное воздействие на спектр оказывает учёт {\color{red}поглощения в горе} и {\color{red}узнать}, как он будет взаимодействовать с осцилляциями при {\color{red}погружённом детекторе}. При переходе к другим экспериментам понадобятся спектры для других обсерваторий.}

Рисунок~\ref{Honda11max10E-1+4} представляет зависимость потоков нейтрино от энергии на всём данном авторами промежутке для всех данных зенитных углов. Вероятно, дрожания спектров электронных (анти)нейтрино в области высоких энергий обусловлены низкой статистикой Монте-Карло. Спектры мюонных (анти)нейтрино выраженных дрожаний не демонстрируют. Подробно область высоких энергий показана на Рис.~\ref{Honda11max10E+3+4}.
\begin{figure}[!ht]
\includegraphics[width=0.5\textwidth]{../pictures/H11xen10nE-1+4.eps}
\includegraphics[width=0.5\textwidth]{../pictures/H11xea10nE-1+4.eps}
\includegraphics[width=0.5\textwidth]{../pictures/H11xen10pE-1+4.eps}
\includegraphics[width=0.5\textwidth]{../pictures/H11xea10pE-1+4.eps}
\includegraphics[width=0.5\textwidth]{../pictures/H11xmn10nE-1+4.eps}
\includegraphics[width=0.5\textwidth]{../pictures/H11xma10nE-1+4.eps}
\includegraphics[width=0.5\textwidth]{../pictures/H11xmn10pE-1+4.eps}
\includegraphics[width=0.5\textwidth]{../pictures/H11xma10pE-1+4.eps}
\caption{Энергетические спектры нейтрино Honda11~\cite{Honda:2011nf} в максимуме солнечной активности на всём данном авторами промежутке для всех данных зенитных углов. \textit{Построено с помощью скрипта \texttt{HondasE.c} при включении длинных опций рисования по данным таблиц \texttt{H11xsp*10*.dat}, насчитанных программой \texttt{spHonda} при включении средних опций \texttt{NC\_ex} и \texttt{Carr(NC\_ex)}.} При выбранном масштабе невозможно изобразить разницу между $\theta$ и $\pi-\theta$. Видны дрожания спектров электронных (анти)нейтрино при высоких энергиях}
\label{Honda11max10E-1+4}
\end{figure}

\clearpage
\begin{figure}[!ht]
\includegraphics[width=0.245\textwidth]{../pictures/H11xen10nE+3+4.eps}
\includegraphics[width=0.245\textwidth]{../pictures/H11xen10pE+3+4.eps}
\includegraphics[width=0.245\textwidth]{../pictures/H11xea10nE+3+4.eps}
\includegraphics[width=0.245\textwidth]{../pictures/H11xea10pE+3+4.eps}
\includegraphics[width=0.245\textwidth]{../pictures/H11xmn10nE+3+4.eps}
\includegraphics[width=0.245\textwidth]{../pictures/H11xmn10pE+3+4.eps}
\includegraphics[width=0.245\textwidth]{../pictures/H11xma10nE+3+4.eps}
\includegraphics[width=0.245\textwidth]{../pictures/H11xma10pE+3+4.eps}
\caption{Область высоких энергий Рис.~\ref{Honda11max10E-1+4} подробно. Видны выраженные дрожания спектров электронных (анти)нейтрино, спектры мюонных (анти)нейтрино гладкие}
\label{Honda11max10E+3+4}
\end{figure}

На Рис.~\ref{Honda11max6E-1+0.5} показано различие спектров при $\theta$ и $\pi-\theta$. Honda соединяет 3D-расчёт с 1D при $E_{\nu}=32$~ГэВ, однако пишет, что это заметно только в зависимости от азимутального угла, а усреднённые по азимутальным углам спектры перестают отличаться в 3D и 1D при $E_{\nu}=3.2$~ГэВ~\cite{Honda:2011nf}.
\begin{figure}[!ht]
\includegraphics[width=0.245\textwidth]{../pictures/H11xen6E-1+0.5.eps}
\includegraphics[width=0.245\textwidth]{../pictures/H11xea6E-1+0.5.eps}
\includegraphics[width=0.245\textwidth]{../pictures/H11xmn6E-1+0.5.eps}
\includegraphics[width=0.245\textwidth]{../pictures/H11xma6E-1+0.5.eps}
\caption{Энергетические спектры нейтрино Honda11~\cite{Honda:2011nf} в максимуме солнечной активности в области низких энергий для избранных пар углов $\theta$ и $\pi-\theta$. \textit{Построено с помощью скрипта \texttt{HondasE.c} при включении коротких опций рисования по данным таблиц \texttt{H11xsp*6.dat}, насчитанных программой \texttt{spHonda} при включении нижних опций \texttt{NC\_ex} и \texttt{Carr(NC\_ex)}.} При энергиях выше $5$~ГэВ разница между спектрами при $\theta$ и $\pi-\theta$ становится незаметной}
\label{Honda11max6E-1+0.5}
\end{figure}

\clearpage
Рисунок~\ref{Honda11maxCE-1+4} изображает зенитно-угловые распределения потоков нейтрино. \textsf{Отношение потоков нейтрино к зенитным при высоких энергиях из общих соображений должно возрастать монотонно. Чем ближе к горизонтали направление движения мезонов и мюонов в атмосфере, тем дольше они не попадут в плотные слои, и тем с большей вероятностью они распадутся до взаимодействия. Из-за замедления времени жизни частицы пропорционально лоренц-фактору этот эффект тем сильнее, чем больше энергия. Эти рассуждения не имеют отношения к низким энергиям, где нейтрино не сохраняют направлений движения исходных частиц, а также к <<прямым>> нейтрино, поскольку время жизни их <<родителей>> крайне мало и они быстро распадаются до взаимодействия даже при очень высоких энергиях.} 
\begin{figure}[!ht]
\includegraphics[width=0.5\textwidth]{../pictures/H11xenCE-1+4.eps}
\includegraphics[width=0.5\textwidth]{../pictures/H11xeaCE-1+4.eps}
\includegraphics[width=0.5\textwidth]{../pictures/H11xmnCE-1+4.eps}
\includegraphics[width=0.5\textwidth]{../pictures/H11xmaCE-1+4.eps}
\caption{Зенитно-угловые распределения потоков нейтрино Honda11~\cite{Honda:2011nf} в максимуме солнечной активности для избранных энергий из всего данного авторами промежутка. \textit{Построено с помощью скрипта \texttt{HondapC.c} по данным таблиц \texttt{H11xza*.dat}, насчитанных программой \texttt{spHonda}.} Видно, что отношение потоков электронных антинейтрино к зенитным при высоких энергиях не возрастает монотонно: при $E_{\nu}=10^{3}$~ГэВ оно больше, чем при $E_{\nu}=10^{4}$~ГэВ}
\label{Honda11maxCE-1+4}
\end{figure}

Но, как показывает Рис.~\ref{Honda11maxrtzE-1+4}, в случае электронных (анти)нейтрино монотонного роста не происходит. \textsf{Это может объясняться следующим. В отличие от пионов и каонов, мюон при взаимодействии не исчезает, а просто теряет энергию: двигаясь по вертикали высокоэнергетичный мюон до распада может потерять до ГэВа, а по горизонтали --- десятки ГэВ. Распад такого мюона будет давать вклад в поток нейтрино существенно меньших энергий, чем распад не терявшего энергию, что приведёт к понижению максимума. {\color{brown}Учёт поляризации и деполяризации из-за потерь энергии мюона приведет к дополнительной перекачке энергии электронных нейтрино.} В потоки мюонных нейтрино основной вклад при высоких энергиях вносят распады мезонов, так что для них этот эффект не очень важен. Также на монотонность могут влиять флуктуации энергетических потерь, но, по интуитивным оценкам, не сильно.} Однако выше упомянутые дрожания, обусловленные, по-видимому, статистическими флуктуациями, {\color{red}больше} указанных эффектов, что приводит к недоверию к спектру Honda11 в области высоких энергий. {\color{blue}К доверию принят интервал $E_{\nu}\in[10^{-1},10^{3}]$~ГэВ.} {\color{MYgreen}Необходимо ограничить использование спектра Honda11 в расчётах принятой областью. Соответственно, нужно оценить вклад в потоки лептонов, который дает ненадежный <<хвост>>.} В квазиупругие события вклад нейтрино с энергиями больше 1~ТэВ не должен быть существен.
\begin{figure}[!ht]
\includegraphics[width=0.5\textwidth]{../pictures/H11xenrtzE-1+4.eps}
\includegraphics[width=0.5\textwidth]{../pictures/H11xeartzE-1+4.eps}
\includegraphics[width=0.5\textwidth]{../pictures/H11xmnrtzE-1+4.eps}
\includegraphics[width=0.5\textwidth]{../pictures/H11xmartzE-1+4.eps}
\caption{Отношения потоков нейтрино Honda11~\cite{Honda:2011nf} к зенитным в максимуме солнечной активности для избранных углов на всём данном авторами промежутке. \textit{Построено с помощью скрипта \texttt{HondazC.c} по данным таблиц \texttt{H11xsp*13.dat}, насчитанных программой \texttt{spHonda} при включении предпоследних опций \texttt{NC\_ex} и \texttt{Carr(NC\_ex)}.} {\color{magenta}Неподходящий для печати формат.} Видно, что отношения потоков электронных (анти)нейтрино к зенитным при высоких энергиях не возрастают монотонно. У верхней границы интервала энергий, около 5--6~ТэВ, заметны небольшие дрожания графиков мюонных (анти)нейтрино. Заметные дрожания графиков электронных (анти)нейтрино начинаются около 30--40~ГэВ. У верхней границы интервала энергий дрожания графиков электронных антинейтрино огромны}
\label{Honda11maxrtzE-1+4}
\end{figure}

Рисунки~\ref{Honda11maxsE-1+3} и \ref{Honda11maxE-1+3} изображают спектры в принятой к доверию области энергий нейтрино. На Рис.~\ref{Honda11maxCE-1+3} среди зенитно-угловых распределений не показано распределение для энергии $E_{\nu}=10^{4}$~ГэВ, выходящее за эту область.
\begin{figure}[!ht]
\includegraphics[width=0.5\textwidth]{../pictures/H11xenE-1+3.eps}
\includegraphics[width=0.5\textwidth]{../pictures/H11xeaE-1+3.eps}
\includegraphics[width=0.5\textwidth]{../pictures/H11xmnE-1+3.eps}
\includegraphics[width=0.5\textwidth]{../pictures/H11xmaE-1+3.eps}
\caption{Энергетические спектры нейтрино Honda11~\cite{Honda:2011nf} в максимуме солнечной активности в принятой к доверию области энергий для избранных пар углов $\theta$ и $\pi-\theta$. \textit{Построено с помощью скрипта \texttt{HondapE.c} по данным таблиц \texttt{H11xsp*13.dat}, насчитанных программой \texttt{spHonda} при включении предпоследних опций \texttt{NC\_ex} и \texttt{Carr(NC\_ex)}. {\color{magenta}Неподходящий для печати формат}}}
\label{Honda11maxsE-1+3}
\end{figure}
\begin{figure}[!ht]
\begin{center}
\includegraphics[width=0.7\textwidth]{../pictures/H11xCE-1+3.eps}
\end{center}
\caption{Зенитно-угловые распределения потоков нейтрино Honda11~\cite{Honda:2011nf} в максимуме солнечной активности для избранных энергий из принятой к доверию области энергий. \textit{Построено с помощью скрипта \texttt{HondaxC.c} по данным таблиц \texttt{H11xza*.dat}, насчитанных программой \texttt{spHonda}}}
\label{Honda11maxCE-1+3}
\end{figure}

\clearpage
\begin{figure}[!ht]
\begin{center}
\includegraphics[width=0.7\textwidth]{../pictures/H11xE-1+3.eps}
\end{center}
\caption{Диапазоны изменения энергетических спектров нейтрино Honda11~\cite{Honda:2011nf} в максимуме солнечной активности в принятой к доверию области энергий при пробегании всего интервала значений зенитного угла. \textit{Построено с помощью скрипта \texttt{HondaxE.c} по данным таблиц \texttt{H11xsp*.dat}, насчитанных программой \texttt{spHonda} при включении верхних опций \texttt{NC\_ex} и \texttt{Carr(NC\_ex)}}}
\label{Honda11maxE-1+3}
\end{figure}

\subsection{Высокоэнергетичные спектры}
\subsubsection{CORT}
\subsubsection{Синеговский}

\section{Спектры ускорительных нейтрино}
\subsection{NO$\nu$A}
